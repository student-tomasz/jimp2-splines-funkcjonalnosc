\documentclass[10pt,a4paper]{article}
\usepackage[a4paper,top=3cm,bottom=3cm]{geometry}

\usepackage{polski}
\usepackage{xltxtra}
\usepackage{relsize}
\usepackage{alltt}
\usepackage{fancyvrb}
\usepackage[pdfborder={0 0 0}]{hyperref}

\defaultfontfeatures{Mapping=tex-text}
\setromanfont{Charis SIL}
%\setsansfont[Scale=MatchLowercase]{Gill Sans}
\setmonofont[Scale=MatchLowercase]{Menlo}
\linespread{1.25}

\DefineVerbatimEnvironment%
  {SmallVerbatim}%
  {Verbatim}{fontsize=\relsize{-0.5},numbers=left,numbersep=-10pt,frame=lines,tabsize=4}

\newcommand{\prog}[1]{\texttt{#1}}
\newcommand{\flag}[1]{\textbf{\prog{#1}}}

\begin{document}

%%fakesection{Tytuł}
\title{ 
  Interpolacja funkcjami sklejanymi\\
  {\normalsize Specyfikacja funkcjonalna projektu nr 1}\\\vspace{-12pt}
  {\normalsize z przedmiotu \emph{Języki i metody programowania 2}}
}
\author{
  Tomasz Cudziło\\
  {\small EE PW, 211552}
}
\date{\today}
\maketitle

\section*{Zadanie}
\label{sec:zadanie}

Napisać program wyznaczający współczynniki funkcji sklejanych trzeciego stopnia
aproksymujących zadany ciąg danych pomiarowych.

\vspace{24pt}

\section{Opis}
\label{sec:opis}

Program na podstawie danych pomiarowych wyznacza współczynniki funkcji
sklejanych je aproksymujących.

Opcjonalnie program tworzy plik dla pakietu \prog{gnuplot}, pozwalający
narysować wykres z~podanymi punktami oraz wyznaczonymi funkcjami sklejanymi.

Program działa jedynie w~wersji tekstowej, nie prowadzi dialogu
z~użytkownikiem. Wymaga podania, jako argumentu linii poleceń, ścieżkę
(względną lub absolutną) do pliku z~danymi wejściowymi, oraz ewentualny ciąg
flag i~ich argumentów opisanych w~sekcji \ref{sec:flagi}.

\section{Obsługa}

\begin{alltt}
\textbf{spline} [\textsl{flags}] [\flag{-o|--out} \textsl{output\_file}] \textsl{source\_file}
\textbf{spline} [\textsl{flags}] \flag{-o|--out} \textsl{output\_file} < \textsl{source\_file}
\end{alltt}

\section{Pliki}
\label{sec:pliki}

\subsection{Plik z~danymi wejściowymi}
\label{sec:plik_wejsciowy}

W~podstawowym przypadku użytkownik musi przekazać ścieżkę do pliku z~danymi.
Jeśli ścieżka do pliku z~danymi jest niewłaściwa lub użytkownik nie ma praw
odczytu do pliku, wtedy program kończy działanie z~adnotacją w~logu.

Jeśli plik wejściowy nie zostanie podany, dane wejściowe wczytywane są
z~wejścia standardowego i~wymagane jest podanie ścieżki do pliku wyjściowego
poprzez flagę \flag{-o|--out}.

W~pliku z~danymi w~jednej linii, powinna znajdować się para liczb oddzielonych
od siebie dowolną liczbą spacji lub znaków tabulacji. Liczby powinny być
sformatowane do postaci rozumianej przez standardową bibliotekę jezyka C.
Linie zaczynające się od znaku \prog{\#} są ignorowane.

Przykład poprawnego pliku z~danymi wejściowymi:
\vspace{-12pt}
\begin{SmallVerbatim}
    # example_input_file
    # x y
    12     -0.9
    7.12e1  3.2e-1  
    92.0    0.04
\end{SmallVerbatim}

W~przypadku niepoprawnie sformatowanych danych, niezależnie od ich źródła,
program przerywa pracę i~drukuje komunikat w~logu.

\subsection{Plik wyjściowy}
\label{sec:plik_wyjsciowy}

Plik wynikowy ma tę samą ścieżkę/nazwę co plik z~danymi z~dopisanym
postfixem~\prog{.out}. Jeśli isnieje już plik o~nazwie pliku wyjściowego,
program przerywa działanie. Jeśli zostanie podana flaga \flag{-f|--force}
plik ten zostanie nadpisany. Plik utworzony przez program\footnote{Funkcje
w~przykładzie są zupełnie przypadkowe.}:
\vspace{-12pt}
\begin{SmallVerbatim}
    # created at: 01/03/2011 00:39:42
    #       file: example_input_file.out
    # [x=x_{n1}:x_{n2}] fun_n(x) = a_{n1}*x**3 + a_{n2}*x**2 + a_{n3}*x + a_{n4}
    [x=12:71.2] fun1(x) = 4*x**3 + 4.4*x**2 - 4.44*x - 5.4
    [x=71.2:92.0] fun2(x) = 5.4*x**3 + 4.44*x**2 - 4.4*x + 4
\end{SmallVerbatim}

\subsection{Plik dla \texttt{gnuplot}}
\label{sec:plik_gnuplot}

Gdy zostanie podana flaga \flag{-g|--gnuplot} program tworzy plik, o~tej samej
nazwie co plik wyjściowy z~rozszerzeniem \prog{.plt}. Jeśli plik o~danej nazwie
już istnieje, zostaje on nadpisany. Domyślnie, plik ten \emph{nie} jest
tworzony.

Plik \prog{.plt} jednoznacznie opisuje wykres z~każdą z~wyznaczonych funkcji
na jej przedziale --- przedziały te nie pokrywają się.
FAQ\footnote{\url{http://www.gnuplot.info/faq/faq.html\#SECTION00081000000000000000}}
\texttt{gnuplota} posiada przykład, który pasuje tutaj doskonale:
\vspace{-12pt}
\begin{SmallVerbatim}
    # created at: 01/03/2011 00:39:42
    #       file: example_input_file.plt
    set parametric
    x11=12
    x12=71.2
    x21=71.2
    x22=92.0
    x1(t) = x11+(x12-x11)*t
    x2(t) = x21+(x22-x21)*t
    fun1(x) = 4*x**3 + 4.4*x**2 - 4.44*x - 5.4
    fun2(x) = 5.4*x**3 + 4.44*x**2 - 4.4*x + 4
    plot [t=0:1]\
      x1(t), fun1(x1(t)),\
      x2(t), fun2(x2(t))
\end{SmallVerbatim}

Każda para podanych programowi wartości $(x_{ij}, x_{ij+1})$ jest mapowana na
przedziale $x_{i}$ ponieważ $t \in [0,1]$, na nim też rysowana jest funkcja
z~pliku wyjściowego. Jeśli znajdę sposób na wyraźniejsze zaznaczenie punktów
podanych w~pliku wejściowym dodam instrukcje do pliku \prog{.plt}.

\subsection{Plik z~logiem}
\label{sec:plik_log}

Plik tekstowy z~logiem jest tworzony analogicznie do pliku wynikowego
z~zakończeniem \prog{.log}. Podanie flagi \flag{-q|--quiet} zapobiega tworzeniu
tego pliku, w~przeciwnym przypadku, jeśli istnieje plik o~danej nazwie, zostaje
on nadpisany.

Dokładna zawartość pliku z~logiem silnie zależy od implementacji projektu.
Ogólnie w~każdej linii będzie nazwa modułu/funckji, lub nazwa pliku źródłowego
i~numer linii kodu tworzących wpis do logu, oraz przekazywana wiadomość. Plik
\prog{.log} będzie zakończony wiadomością o~pomyślnym zakończeniu lub opisem
napotkanego błędu.
\vspace{-12pt}
\begin{SmallVerbatim}
    # created at: 01/03/2011 00:39:42
    #       file: example_input_file.log
    parser.c:80: parsing some points
    spline.c:92: received ranges
    main.c:80: omfg, something bad happened
    error: interpolation aborted because failwhale ate my pencil
\end{SmallVerbatim}

\section{Flagi}
\label{sec:flagi}

\begin{description}
  \item[\flag{-o}{\tt\textsl{=file}},\quad\flag{--out}{\tt\textsl{=file}}] \hfill \\
    Utwórz plik wyjściowy o~podanej nazwie. Jeśli flaga jest podana wtedy nazwy
    pozostałych plików tworzonych przez program są wzorowane na jej argumencie,
    a~nie na nazwie pliku wejściowego, tzn. dodawane są postfixy \prog{.plt}
    i~\prog{.log}. Jeśli nazwa podanego pliku kończy się na \prog{.out} wtedy
    rozszerzenie to jest pomijane w~nazwach pozostałych plików.
  \item[\flag{-g},\quad\flag{--gnuplot}] \hfill \\
    Razem z~plikem wyjściowym utwórz plik z~instrukcjami dla \prog{gnuplot}.
    Domyślnie wyłączone.
  \item[\flag{-f},\quad\flag{--force}] \hfill \\
    Wymusza nadpisanie pliku, gdy plik wynikowy o~żądanej nazwie już istnieje.
    Domyślnie w~takiej sytuacji program przerywa działanie. Gorzej jak plik
    istnieje, ale użytkownik nie ma praw go ruszać.
  \item[\flag{-q},\quad\flag{--quiet}] \hfill \\
    Nie twórz pliku z~logiem. Domyślnie \prog{.log} jest tworzony.
\end{description}

\end{document}
